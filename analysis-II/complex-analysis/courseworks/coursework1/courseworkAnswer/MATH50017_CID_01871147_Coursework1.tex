\documentclass[12pt]{article}
%% Package imports
\usepackage[utf8]{inputenc}
\usepackage{amsmath}
\usepackage{subcaption}
\usepackage{pgfplotstable}
\usepackage{amsfonts}
\usepackage{amssymb}
\usepackage{graphicx}
\usepackage{physics}
\usepackage[left=2cm,right=2cm,top=2cm,bottom=2cm]{geometry}
\usepackage{multirow}
\usepackage{booktabs}
\usepackage{float}
\usepackage{verbatim}
\usepackage{amsthm}
\usepackage{fancyhdr}
\usepackage[shortlabels]{enumitem}
\renewcommand{\baselinestretch}{1.5}

%% Commands for inserting big braces.
\newcommand\lb{\left\lbrace}
\newcommand\rb{\right\rbrace}

%% Math symbols
\newcommand\Q{\mathbb{Q}}
\newcommand\R{\mathbb{R}}
\newcommand\N{\mathbb{N}}
\newcommand\C{\mathbb{C}}
\newcommand\W{\overline{W}}
\newcommand\intR{\int_{-\infty}^\infty}

\newcommand\st{\text{ such that }}
\parindent 0ex

%% Page style settings
\pagestyle{fancy}
\fancyfoot{}
\fancyhead[L]{\slshape{Complex Analysis}}
\fancyhead[R]{\slshape{CID: 01871147}}
\fancyfoot[C]{\thepage}
\begin{document}
\title{Complex Analysis Spring 2022 Coursework}
\date{\today}
\author{Szymon Kubica, CID: 01871147}
\maketitle

\section*{Q1.}

Since we know that the function $ \text{Log } z $ is holomorphic everywhere in $\C \setminus (-\infty, 0)$, and also that
$e^z$ and $\text{sin } z$ are entire, we may deduce that $\sqrt{z}$ defined as $e^{\frac{1}{2} \text{Log} z}$
is holomorphic everywhere in $\C \setminus (-\infty, 0)$ and so by composition the function $f(z) = 2\text{sin}(\sqrt{z})$
is holomorphic inside of that set. Now since $i\frac{\pi^2}{2} \in \C \setminus (-\infty, 0)$ we can apply the
theorem in the notes:
\[f'(z_0) = \frac{\partial f}{\partial z} (z_0)\]
And so we may calculate:
\[f'\left(i\frac{\pi^2}{2}\right) = \frac{\text{cos}(\sqrt{i\frac{\pi^2}{2}})}{\sqrt{i\frac{\pi^2}{2}}}\]
Now let us compute
\[\sqrt{i\frac{\pi^2}{2}} = e^{\frac{1}{2} \text{Log} (i\frac{\pi^2}{2})} = e^{\frac{1}{2} \left( \text{log} (\frac{\pi^2}{2}) + i\frac{\pi}{2}\right)} = \frac{\pi}{\sqrt{2}} e^{i\frac{\pi}{4}} = \frac{\pi}{\sqrt{2}}\left(\frac{1}{\sqrt{2}} + i\frac{1}{\sqrt{2}}\right) =  \frac{\pi}{2}(1 + i). \]
Our expression becomes:
\[f'\left(i\frac{\pi^2}{2}\right) = \frac{\text{cos}(\frac{\pi}{2}(1 + i))}{\frac{\pi}{2}(1 + i)}.\]
We can simplify it using the definition of complex cosine in the numerator and the multiplication by the conjugate in
the denominator.

\[ = \frac{1}{2} \frac{\left(e^{(i\frac{\pi}{2}(1 + i))} + e^{(-i\frac{\pi}{2}(1 + i))}\right)(1 - i)}{\frac{\pi}{2}(1 + i)(1 - i)}
= \frac{1}{\pi} \frac{\left(e^{(\frac{\pi}{2}(i - 1))} + e^{(-\frac{\pi}{2}(i - 1))}\right)(1 - i)}{2}
= \frac{1}{2\pi} \left(e^{(\frac{\pi}{2}(i - 1))} + e^{(-\frac{\pi}{2}(i - 1))}\right)(1 - i)\]
If we now simplify it even further, we get:

\[= \frac{1}{2\pi} \left(e^{-\frac{\pi}{2}}\left(\text{cos}(\frac{\pi}{2}) + i\text{sin}(\frac{\pi}{2})\right) + e^{\frac{\pi}{2}}\left(\text{cos}(\frac{\pi}{2}) - i\text{sin}(\frac{\pi}{2})\right)\right)(1 - i).\]
Now by evaluating sin and cos in the expression above, we get:
\[= \frac{1}{2\pi} \left(e^{-\frac{\pi}{2}}i - e^{\frac{\pi}{2}}i\right)(1 - i) = \frac{e^{-\frac{\pi}{2}} - e^{\frac{\pi}{2}}}{2\pi} + i\frac{e^{-\frac{\pi}{2}} - e^{\frac{\pi}{2}}}{2\pi}.\]
If we note that $ \text{sinh } x = \frac{e^x - e^{-x}}{2} $ we can rewrite the expression above as:
\[-\frac{\text{sinh}(\frac{\pi}{2})}{\pi} - i \frac{\text{sinh}(\frac{\pi}{2})}{\pi}\]

\section*{Q2.}
\subsection*{(a)}
Consider the following parameterisation of $\gamma$:
\[\gamma := \{ z=\rho e^{i\theta} | \theta \in [0, 2\pi] \}\]
Now since $ z = z(r, \theta) $ using the total differentiation, we obtain:

\[
	\frac{\text{d} z}{\text{d} \theta} = \frac{\partial z}{\partial \theta} + \frac{\partial z}{\partial r} \frac{\text{d} r}{\text{d} \theta}
.\]

Since $\gamma$ is a circle, the radius is constant and so $\frac{\text{d} r}{\text{d} \theta}$ is 0.
Therefore the expression above becomes:
\[
	\frac{\text{d} z}{\text{d} \theta} = \frac{\partial z}{\partial \theta}
	= \frac{\partial  }{\partial \theta} \left(\rho e^{i\theta}\right)
	= i\rho e^{i\theta}
.\]
Hence we can deduce that the following holds:
\begin{equation}
	\text{d}z = i\rho e^{i\theta} \text{d}\theta
.\end{equation}
Now let us consider $|\text{d}z|$
\[
	|\text{d}z| = |i\rho e^{i\theta} \text{d} \theta|
.\]

Since we have parameterised $\theta$ to range from $0 \text{ to } 2\pi$ we know that $\text{d} \theta$ is real and positive and hence we can simplify:
\[
|\text{d}z| = |i\rho e^{i\theta}|\text{d} \theta = |i| |\rho e^{i\theta}| \text{d}\theta = \rho \text{d}\theta
.\]

Now from (1) we can also deduce that $\text{d}\theta = \dfrac{\text{d}z}{i \rho e^{i\theta}} = \dfrac{\text{d}z}{iz}$
Combining the above two together, we deduce:
\[
|\text{d}z| = \rho \frac{\text{d}z}{iz} =  - i\rho \frac{\text{d}z}{z}
.\]

\subsection*{(b)}
In order to compute the integral in question we first substitute the result from part (a).
\[
	\oint_\gamma \frac{|\text{d}z|}{|z - a|^2} = - \oint_\gamma \frac{\rho i}{z |z - a|^2} \text{d} z
.\]

We can apply the definition of the complex norm to the expression in the denominator to obtain:
\[
	= - \oint_\gamma \frac{\rho i}{z (z - a) (\overline{z - a})} \text{d} z
.\]
By the properties of the conjugate, it becomes:
\[
= - \oint_\gamma \frac{\rho i}{z (z - a) (\bar{z} - \bar{a})} \text{d} z
.\]

Now if we note that $\bar{z} = \frac{|z|^2}{z}$ we can rewrite the integral as:
\[
	= - \frac{\rho i}{\bar{a}}\oint_\gamma \frac{\text{d}z}{(z - a) (\frac{\rho^2}{\bar{a}} - z)}
.\]

In order to compute the integral above, we need to consider two cases, the first one being when $a$ is interior to
 $\gamma$ and the second one when  it is outside of the region enclosed by $\gamma$.
 Observe that in the first case, we clearly have $|a| < \rho$ and $|a| = |\bar{a}|$ hence for all  $z$ on and inside  $\gamma$ we have:
\[
 \left|\frac{\rho^2}{\bar{a}} - z\right| \ge \left|\frac{\rho^2}{\bar{a}}\right| - |z| =
	 \frac{\rho^2}{|a|} - |z| > \rho - |z| \geq \rho - \rho = 0
.\]

The last transition in the expression above is because if $z$ is on or inside $\gamma$ then necessarily $|z| \leq \rho$
Therefore we have managed to show that the norm of $(\frac{\rho^2}{\bar{a}} - z)$ is greater than zero for all $z$ on
and inside $\gamma$ and therefore it is never zero, and so we may deduce that  $\dfrac{1}{\frac{\rho^2}{\bar{a}} - z}$
is holomorphic on and inside $\gamma$ and so we may apply the Cauchy's integral formula to evaluate:
\[
	- \frac{\rho i}{\bar{a}}\oint_\gamma \frac{\text{d}z}{(z - a) (\frac{\rho^2}{\bar{a}} - z)} =
	- \frac{\rho i}{\bar{a}}\left(2\pi i \frac{1}{\frac{\rho^2}{\bar{a}} - a} \right)
	= \frac{2 \pi \rho}{\rho^2 - |a|^2}
.\]

Now in the case when $a$ is outside the region enclosed by $\gamma$, then clearly $(a - z)$ is not zero for all points
	$z$ inside and on  $\gamma$. Therefore  $\frac{1}{z - a}$ is holomorphic on and inside $\gamma$.
Now we just need to show that  $\frac{\rho^2}{\bar{a}}$ is interior to $\gamma$ in order to be able to apply the
Cauchy's integral formula. Observe that since $a$ is outside of the region enclosed by  $\gamma$, we have
$|\bar{a}| > \rho $ therefore we may deduce that $\frac{\rho^2}{|\bar{a}|} < \rho$ and so necessarily
$\frac{\rho^2}{\bar{a}}$ is interior to $\gamma$ and so we may apply the Cauchy's integral formula by letting  $f(z) = \frac{1}{z - a}$ around the point $\frac{\rho^2}{\bar{a}}$. Hence we obtain:
\[
	- \frac{\rho i}{\bar{a}}\oint_\gamma \frac{\text{d}z}{(z - a) (\frac{\rho^2}{\bar{a}} - z)} =
	 \frac{\rho i}{\bar{a}}\oint_\gamma \frac{\text{d}z}{(z - a) (z - \frac{\rho^2}{\bar{a}})} =
	 \frac{\rho i}{\bar{a}}\left(2\pi i \frac{1}{\frac{\rho^2}{\bar{a}} - a} \right) =
	 - \frac{2 \pi \rho}{\rho^2 - |a|^2}
.\]

\section*{Q3.}
\subsection*{(a)}
In order to show the identities required in the question let us consider the following series for $0 < \theta < 2\pi$
\begin{equation}
	 \sum^{n}_{k = 0} e^{ik\theta} = \sum^{n}_{k = 0} \text{cos}(k\theta) + i\sum^{n}_{k = 0} \text{sin}(k\theta)
 .\end{equation}

If we consider the left-hand side of the equation above as a geometric series, we obtain:
\[
	\sum^{n}_{k = 0} e^{ik\theta} = \frac{1 - (e^{i\theta})^{n+1}}{1 - (e^{i\theta})}
.\]

We can now rewrite it and remove the imaginary numbers from the denominator, by multiplying by the conjugate:
\[
	= \frac{1 - \text{cos}((n+1)\theta) - i\text{sin}((n+1)\theta)}{(1 - \text{cos}\theta) - i\text{sin}\theta}
	\frac{(1 - \text{cos}\theta) + i\text{sin}\theta}{(1 - \text{cos}\theta) + i\text{sin}\theta}
.\]

After multiplying out and simplifying terms using the following trigonometric identities:
\[
	\cos\theta\cos((n+1)\theta) + \sin\theta\sin((n+1)\theta) = \cos((n+1)\theta - \theta) = \cos(n\theta)
,\]
\[
	\sin((n+1)\theta)\cos\theta - \cos((n+1)\theta)\sin\theta   = \sin((n+1)\theta - \theta) = \sin(n\theta)
,\]
we get:
\begin{equation}
	= \frac{1 - \cos\theta + \cos(n\theta) - \cos((n+1)\theta)  + i(\sin\theta + \sin(n\theta) - \sin((n+1)\theta)}{2(1 - \cos\theta)}
.\end{equation}
Let us now consider the real part of the expression above. Observe that we can factor out $\frac{1}{2}$ :
\begin{equation}
	\frac{1 - \cos\theta + \cos(n\theta) - \cos((n+1)\theta)}{2(1 - \cos\theta)} = \frac{1}{2} + \frac{\cos(n\theta) - \cos((n+1)\theta)}{1 - \cos\theta}
.\end{equation}

Now let us use the following trigonometric identities:
\[
	\cos(2\theta) = 1 - 2\sin^2\theta \implies 1 - \cos\theta = 2 \sin^2\left(\frac{\theta}{2}\right)
.\]
\[
	\cos((n+1)\theta) = \cos(\left(n+\frac{1}{2}\right)\theta + \frac{\theta}{2}) =
	\cos(\left(n+\frac{1}{2}\right)\theta) \cos(\frac{\theta}{2}) - \sin(\left(n+\frac{1}{2}\right)\theta) \sin(\frac{\theta}{2})
.\]
\[
	\cos(n\theta) = \cos(\left(n+\frac{1}{2}\right)\theta - \frac{\theta}{2}) =
	\cos(\left(n+\frac{1}{2}\right)\theta) \cos(\frac{\theta}{2}) + \sin(\left(n+\frac{1}{2}\right)\theta) \sin(\frac{\theta}{2})
.\]

If we apply the above to (4), we obtain:
\[
	= \frac{2\sin(\left(n+\frac{1}{2}\right)\theta) \sin(\frac{\theta}{2})}{4\sin^2\left(\frac{\theta}{2}\right)} =
	\frac{\sin(\left(n+\frac{1}{2}\right)\theta)}{2\sin\left(\frac{\theta}{2}\right)}
.\]
And now by (2) we know that the expression above being the real part of the geometric sum has to be equal to
$\sum^{n}_{n = 0} \text{cos}(n\theta)$

For the second identity we need to consider the imaginary part of the expression (3).
\[
	\frac{(\sin\theta + \sin(n\theta) - \sin((n+1)\theta)}{2(1 - \cos\theta)}
.\]

For the denominator we use the same identity as before and for the numerator we use the following two identities
\[
	\sin((n+1)\theta) = \sin(\left(n+\frac{1}{2}\right)\theta + \frac{\theta}{2}) =
	\sin(\left(n+\frac{1}{2}\right)\theta) \cos(\frac{\theta}{2}) + \cos(\left(n+\frac{1}{2}\right)\theta) \sin(\frac{\theta}{2})
.\]
\[
	\sin(n\theta) = \sin(\left(n+\frac{1}{2}\right)\theta - \frac{\theta}{2}) =
	\sin(\left(n+\frac{1}{2}\right)\theta) \cos(\frac{\theta}{2}) - \cos(\left(n+\frac{1}{2}\right)\theta) \sin(\frac{\theta}{2})
.\]

Applying those yields:
\[
	\frac{(\sin\theta  - 2\cos((n+\frac{1}{2})\theta)\sin(\frac{\theta}{2})}{4\sin^2(\frac{\theta}{2})} =
	\frac{\sin\theta}{4\sin^2(\frac{\theta}{2})} -\frac{\cos((n+\frac{1}{2})\theta)}{2\sin(\frac{\theta}{2})}
.\]
Now since $\sin\theta = 2\sin(\frac{\theta}{2})\cos(\frac{\theta}{2})$, we obtain
\[
	\frac{1}{2}\cot(\frac{\theta}{2}) -\frac{\cos((n+\frac{1}{2})\theta)}{2\sin(\frac{\theta}{2})}
.\]

And we deduce that it must be equal to the imaginary part of (2) which we had to show.
\subsection*{(b)}
In order to find the Taylor series in question, let us first split $f(z)$ using partial fractions
\[
	\frac{1}{(z+1)(z+2)} = \frac{1}{(z+1)} - \frac{1}{(z+2)}
.\]
Note that now we can compute the $n$-th derivative of $f(z)$
 \[
	 f^{(n)} = (-1)^{n} n! \left[ \frac{1}{(z+1)^{n+1}} - \frac{1}{(z+2)^{n+1}} \right]
.\]

Hence the Taylor series of $f(z)$ is given by
\[
	f(z) = \sum^{\infty}_{n=0} \frac{(-1)^{n} n! \left[ \frac{1}{(i+1)^{n+1}} - \frac{1}{(i+2)^{n+1}} \right]}{n!} (z - i)^{n}
.\]

That in turn simplifies into
\[
	f(z) = \sum^{\infty}_{n=0} (-1)^{n}\left[ \frac{1}{(i+1)^{n+1}} - \frac{1}{(i+2)^{n+1}} \right](z - i)^{n}
.\]

The radius of convergence of that expansion is given by the distance from $i$ to the closest point where  $f(z)$ is
not holomorphic. Clearly,  $f(z)$ is not holomorphic at  $-1 \text{ and } -2$ and so the closest to $i$ of the two is
$-1$. Hence, as  $|i - (-1)| = \sqrt{2}$  we deduce that the disc of convergence is given by
\[
	|z - i| < \sqrt{2}
.\]

\section*{Q4.}
First observe that for all $n \in \N$ we have:
\[
	\sqrt{\frac{n}{2 + n}} < 1
.\]
That is because, clearly $	\frac{n}{2 + n} < \frac{n}{n} $ and the square root is monotone increasing.
Now for all $n \in \N$ define
\[
	\gamma_n = \{z = re^{i\theta} | r = \sqrt{\frac{n}{2 + n}} \text{ and } \theta \in [0, 2\pi]\}
.\]
Clearly, each one of those paths is contained in $\mathbb{D}$ and so we deduce that for all $n \in \N$, $f(z)$ is
holomorphic on and inside $\gamma_n$, hence we may apply the Cauchy's integral formula around 0:
\[
	|f^{(n)}(0)| = \left|\frac{n !}{2\pi i}\oint_{\gamma_n} \frac{f(\eta)}{(\eta)^{n+1}}\text{d}\eta\right|
.\]

By using the properties of the complex norm, the M-L inequality, and the fact that on
$\gamma_n$ we have  $|\eta| = r = \sqrt{\frac{n}{2 + n}}$ we may bound the norm of the integral above.

\[
	\left|\frac{n !}{2\pi i}\oint_{\gamma_n} \frac{f(\eta)}{(\eta)^{n+1}}\text{d}\eta\right| \leq
	\frac{n !}{2\pi} \sup_{\eta \in \gamma_n} \left|\frac{f(\eta)}{(\eta)^{n+1}}\right|2\pi r =
	n! \sup_{\eta \in \gamma_n} \frac{|f(\eta)|}{r^{n}}
.\]

Now by the given assumption for all $\eta$ in $\gamma_n$ we have
$|f(\eta)| \leq \frac{1}{1 - |\eta|^2} = \frac{1}{1 - r^2}$
And so we get
\[
	n! \sup_{\eta \in \gamma_n} \frac{|f(\eta)|}{r^{n}} \leq n!\frac{1}{1 - r^2}\frac{1}{r^n}
.\]

After substituting our defined value of $r = \sqrt{\frac{n}{2 + n}}$ we get
\[
	n!\frac{1}{1 - r^2}\frac{1}{r^n} = n!\frac{1}{1 - \frac{n}{2+n}}\left(\frac{2+n}{n}\right)^{\frac{n}{2}} =
	n!\frac{2 + n}{2}\left(\frac{2+n}{n}\right)^{\frac{n}{2}} =
	\frac{n!(2+n)^{\frac{(2+n)}{2}}}{2n^{\frac{n}{2}}}
.\]

Which we had to show.

\end{document}
