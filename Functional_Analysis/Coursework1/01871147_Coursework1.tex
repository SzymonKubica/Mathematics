\documentclass[11pt]{article}
%% Package imports
\usepackage[utf8]{inputenc}
\usepackage{amsmath}
\usepackage{subcaption}
\usepackage{amsfonts}
\usepackage{amssymb}
\usepackage{physics}
\usepackage{graphicx}
\usepackage[left=2cm,right=2cm,top=2cm,bottom=2cm]{geometry}
\usepackage{multirow}
\usepackage{booktabs}
\usepackage{float}
\usepackage{verbatim}
\usepackage{amsthm}
\usepackage{fancyhdr}
\usepackage[shortlabels]{enumitem}
\renewcommand{\baselinestretch}{1.5}

%% Commands for inserting big braces.
\newcommand\lb{\left\lbrace}
\newcommand\rb{\right\rbrace}

%% Commands for set such that notation
\newcommand\st{\text{ } | \text{ }}

%% Math symbols
\newcommand\Q{\mathbb{Q}}
\newcommand\R{\mathbb{R}}
\newcommand\N{\mathbb{N}}
\newcommand\C{\mathbb{C}}
\newcommand\l{\mathcal{l}}

\parindent 0ex

%% Page style settings
\pagestyle{fancy}
\fancyfoot{}
\fancyhead[L]{\slshape{Functional Analysis}}
\fancyhead[R]{\slshape{CID: 01871147}}
\fancyfoot[C]{\thepage}
\begin{document}

\title{Functional Analysis Autumn 2022 Coursework 1}
\date{\today}
\author{CID: 01871147}
\maketitle

\section*{Problem Set I}
\subsection*{Exercise I.1.4}

Claim: the following set is not a linear space, because it is not closed under
\( \oplus \).
\[ V := \lb f(z) \text{ - analytic} \left|\text{ } \frac{d^2}{dz^2}f - \frac{d}{dz}f - 2z = 0 \rb\]
Where \(\oplus: V\times V \to V\) is the usual addition of functions:
\[\forall x \in \C \text{ we have } (f \oplus g)(x) = f(x) + g(x)\]

Proof:

Let \(f,g \in V \) be arbitrary, suppose for contradiciton that \(f \oplus g \in V \),
then by definition of V we have that:
\begin{equation}
  \frac{d^2}{dz^2}(f \oplus g) - \frac{d}{dz}(f \oplus g) - 2z = 0
\end{equation}

But now also \(f, g \in V\) so the following holds
\[\frac{d^2}{dz^2}f - \frac{d}{dz}f - 2z = 0\]
\[\frac{d^2}{dz^2}g - \frac{d}{dz}g - 2z = 0\]

If we add the two equations above and exploit linearity of differentiation we get:
\begin{equation}
  \frac{d^2}{dz^2}(f \oplus g) - \frac{d}{dz}(f \oplus g) - 4z = 0
\end{equation}

Now if we subtract the equation (2) from (1) we obtain:
\[\forall z \in \C \text{ }  2z = 0,\]
which is a contradiction and hence we deduce that \(f \oplus g \notin V\) and
so \(V\) is not closed under vector addition and hence is not a linear space.
\hfill\blacksquare

\section*{Problem Set II}
\subsection*{Exercise I.7}

Let $(X, \rho)$ be a metric space, let $\tilde{\rho_z}$ for some $z \in X$
denote the function described in the question:
\[
\begin{split}
  \tilde{\rho_z}: X & \to \R^+  \\
  x & \mapsto \rho(z, x)
\end{split}
\]
We'll show continuity of $\tilde{\rho_z}$ using seqential continuity. Let
$(x_n)_{n \in N}$ be an arbitrary sequence in $X$ convergent to some $x$ with
respect to $\rho$. We need to show that as $n\to\infty$ we have
$\tilde{\rho_z}(x_n) \to \tilde{\rho_z}(x)$
with respect to the Euclidean metric on $\R$. Note that since $\rho$ is a metric
it satisfies the triangle inequality and hence  $\forall n \in \N$
\begin{equation}
\rho(z, x_n) \leq \rho(z, x) + \rho(x, x_n) \implies \rho(z, x_n) - \rho(z, x) \leq \rho(x, x_n)
.\end{equation}

Similarly using the symmetry of a metric:
\begin{equation}
\rho(z, x) \leq \rho(z, x_n) + \rho(x_n, x) \implies \rho(z, x) - \rho(z, x_n) \leq \rho(x, x_n)
.\end{equation}

Hence, by combining the inequalities (3) and (4) above, we have:
\begin{equation}
|\rho(z, x) - \rho(z, x_n)| \leq \rho(x, x_n)
.\end{equation}

Now let $\epsilon > 0$ be arbitrary, since $(x_n)_{n \in \mathbb{N}}$ converges
to $x$ w.r.t  $\rho$ we can pick  $N\in \N$ such that  $\forall n \geq N$ we have
\(
\rho(x, x_n) < \epsilon
.\)

Now also by definition of $\tilde{\rho_z}$ and inequality (5), we have:
\[
|\tilde{\rho_z}(x) - \tilde{\rho_z}(x_n)| =
|\rho(z, x) - \rho(z, x_n)| \leq \rho(x, x_n) < \epsilon
.\]

And so $\tilde{\rho_z}(x_n) \to \tilde{\rho_z}(x)$ as $n \to \infty$ in $\R$
which implies that $\tilde{\rho_z}$
is continuous. \hfill\blacksquare

\section*{Problem Set III}
\subsection*{Exercise I.5 (iii)}

Fix $j \in \N$ and denote $\ell_{p, j} := \lb x \in \ell_p \st x_j = 0 \rb$.
We'll show that $\ell_{p, j}$ is closed in $\ell_p$ and that it doesn't contain
any open ball from $\ell_p$.

Let $(x_n)_{n \in \mathbb{N}}$ be an arbitrary sequence in $\ell_{p, j}$ which
converges to $x \in \ell_p $ in $\ell_p$ w.r.t.  $ \|.\|_p$. We need to show
that $x$ belongs to $\ell_{p, j}$.

Let $\epsion > 0$ be arbitrary, pick $N \in \N$ such that $\forall n \ge N$ we
have $\left(\|x_n - x\|_p\right)^p < \epsilon$. We can do that by convergence of
$(x_n)_{n \in \mathbb{N}}$. Now by the definiton of $ \|.\|_p$. We have:
\[
\sum_{i \in N}^\infty |x_{n_i} - x_i|^p < \epsilon
,\]
where $x_{n_i}$ denotes the $i$-th term of the sequence $x_n$ which is the
 $n$-th entry of  $(x_n)_{n \in \mathbb{N}}$

Now since each $|x_{n_i} - x_i|$ is non-negative, it implies that
\[
  \forall n \ge N \text{ } \forall i \in N \text{ } |x_{n_i} - x_i| < \epsilon
.\]

Now fix $i := j$ (the $j$ that we fixed at the beginning). Note that since each
$x_n$ belongs to $\ell_{p, j}$ we have that $\forall n \in \N \text{ } x_{n_j} = 0 $.
Hence, we may conclude that $\forall n \in \N \text{ } |x_j| < \epsilon$. But that
inequality doesn't depend on $n$, so we have: $|x_j| < \epsilon$. Since $\epsilon$
was arbitrary, we can deduce that:
 \[
   \forall \epsilon > 0 \text{ } |x_j| < \epsilon
.\]
Which implies that $x_j = 0$, and hence $x \in \ell_{p, j}$. Since $(x_n)_{n \in \mathbb{N}}$
was arbitrary, we deduce that $\ell_{p, j}$ is closed.

Now suppose for contradiction $\ell_{p, j}$ contained an arbitrary open ball $B(x, \delta)$
for  $x \in \ell_{p} \text{, } \delta > 0$.

Note that in this case $x \in \ell_{p, j}$ and by definition of an open ball:
\[
\forall y \in \ell_p \text{ } \|x - y \|_p < \delta \implies y \in B(x, \delta) \implies y \in \ell_{p, j}
.\]

If we now define:
\[
y := \begin{cases}
  x_k & k \neq j \\
  \frac{\delta}{2} & k = j
\end{cases}
.\]

And check that indeed $y \in \ell_p$:
\[
\|y\|_p \le \|y - x\|_p + \|x\|_p = \frac{\delta}{2} + \|x\|_p < \infty
,\]
because $x \in \ell_p$. We can observe that:
\[
\|x - y\|_p = \left(\sum_{i \in N}^\infty |x_i - y_i|^p\right)^\frac{1}{p} = \frac{\delta}{2} < \delta
.\]

Which implies that $y \in B(x, \delta) \subset \ell_{p, j}$.
But clearly  $y \notin \ell_{p, j}$, as, by definition, $y_j = \frac{\delta}{2} > 0$
Which is a contradiction hence $\ell_{p, j}$ doesn't contain any open ball
in $\ell_p$. \hfill\blacksquare

\subsection*{Exercise I.5 (iv)}

Let $S := \lb x \in \ell_p \st \forall j \in \N \text{ } |x_j| \le Cj^{-\frac{2}{p}}\rb$
for some $C \in (0, \infty)$, we'll show that $S$ is closed in $\ell_p$.

Let $(x_n)_{n \in \mathbb{N}}$ be an arbitrary sequence in $S$ which
converges to some $x \in \ell_p $ in $\ell_p$ w.r.t.  $ \|.\|_p$. We need to show
that $x$ belongs to $S$. By definition of $S$, take  $j \in \N$ arbitrary. We need
to show that
\[
|x_j| \le Cj^{-\frac{2}{p}}
.\]

Let $\epsilon > 0$ be arbitrary, pick $N \in \N$ such that for all $n \ge N$ we
have
\[
\sum_{i \in N}^\infty |x_{n_i} - x_i|^p < \epsilon^p
.\]
Now in particular, since each term of the sum above is non-negative, we must
have that:
\begin{equation}
  {|x_{n_j} - x_j|}^p < \epsilon^p \text{ and so } {|x_{n_j} - x_j|} < \epsilon
.\end{equation}

Now using the triangle inequality, symmetry, the fact that $x_n \in S$, and inequality (6) we have $\forall n \ge N$:
\[
  |x_j| \le |x_j - x_{n_j}| + |x_{n_j}| = |x_{n_j} - x_j| + |x_{n_j}|
  \le |x_{n_j} - x_j| + Cj^{-\frac{2}{p}} < \epsilon + Cj^{-\frac{2}{p}}
.\]

Hence we obtain that:
\[
  \forall \epsilon > 0 \text{ } |x_j| < \epsilon + Cj^{-\frac{2}{p}}
.\]
Therefore $|x_j| \le Cj^{-\frac{2}{p}}$ and so $x \in S$. Since  $(x_n)_{n \in \mathbb{N}}$
 was arbitrary, we deduce that S is closed in $\ell_p$

\end{document}
